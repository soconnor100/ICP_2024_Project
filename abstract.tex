This report looked at the price of a crude oil barrel every month from January of 1974 to December of 2023. Crude oil prices are somewhat volatile due to how much they are affected by geopolitics. Instead of looking at the raw numbers for every month, the average yearly prices were found and then plotted. This analysis was done through python, with help from the numpy and matplotlib python packages. Python is a strong tool for data analysis due to the variety of available packages, its open-source nature, and the wide number of available resources to help people in their goals for a project. The steps for the analysis here will be explained throughout this text. The report analysis was focused on the causes for the spike in oil prices from 1978-1980, specifically the Iranian Revolution which started in 1978. A simple graph of something like oil prices can seem uninteresting at face value but it can often be representative of larger world events if some time is taken to really dig into the reasons for why changes happen when they do. It also shows how python is not strictly limited to STEM applications, but can also be used to help our understanding of things like history, sociology, and economics.
