As can be seen in the figure \ref{fig:PricePlot}, there was a sharp increase in the price of crude oil in the late 
1970s. \cite{Gross_2019} 
This is specifically known as the 1979 oil crisis and it happened because of the Iranian Revolution which started 
in 1978. Iran was and continues to be one of the largest oil producing countries in the world. 
During the revolution there, oil workers went on strike which led to a 7 percent drop in global oil 
production which made the price of crude oil increase dramatically. From 1979 to 1980, the price of a barrel of 
crude oil had more than doubled which had ripple effects on other countries. \cite{Gross_2019} For example, gasoline is one of the 
most used products that can be made from crude oil so there were gas shortages in the U.S. due to the lower amounts 
of oil. Public fears of completely running out of gas led to somewhat of a panic where people tried to buy more 
gas than they needed which made the problem even worse.


This had further effects on major elections and on the development of oil production. 
Oil was sought out from other regions and production increased until the mid 1980s when the 
trend had completely reversed itself and there was actually an overabundance of crude oil. \cite{Gross_2019}
